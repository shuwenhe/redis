%\documentclass{article}  
\documentclass[12pt,a4paper,openany,fleqn]{book} %声明文档类型

\usepackage{lipsum}
\usepackage[colorlinks]{hyperref}
\renewcommand{\contentsname}{\centerline{目录}}

\usepackage{xeCJK} % 中文
\usepackage{listings} % 插入代码
\usepackage{xcolor} % 代码颜色

%
% 插入代码
%
\lstset{language=C++}%这条命令可以让LaTeX 排版时将C++ 键字突出显示
\lstset{breaklines}%这条命令可以让LaTeX自动将长的代码行换行排版
\lstset{extendedchars=false}% 这一条命令可以解决代码跨页时,章节标题,页眉等汉字不显示的问题
\lstset{
    numbers=left,
    numberstyle= \tiny,
    keywordstyle= \color{ blue!70},
    commentstyle= \color{red!50!green!50!blue!50},
    frame=shadowbox, % 阴影效果
    rulesepcolor= \color{ red!20!green!20!blue!20} ,
    escapeinside=``, % 英文分号中可写入中文
    xleftmargin=2em,xrightmargin=2em, aboveskip=1em,
    framexleftmargin=2em,
}

\setlength\parskip{\baselineskip}% 增加空行%
\setcounter{tocdepth}{4}%控制目录2 级
\setcounter{secnumdepth}{5}%题号显示层级深度

\begin{document}  
\tableofcontents

%
\chapter{DAL到cache}
\section{•}

%
\chapter{mysql主从复制}

%
\chapter{连接池}
\section{pool}
1.max\_line:8最大空闲数\\
2.max\_active:0最大连接数,0不限制\\
3.idle\_time\_out:100最大空闲时间\\
4.c:=pool.Get()//取出一个连接\\
5.c:=pool.Close//关闭

\section{pool.Get}

\section{conn.Close}

\section{conn.Do}

\section{连接池使用案例}
\begin{lstlisting}
var pool *redis.Pool
//当启动程序时就初始化连接池
pool = &redis.Pool{
}
\end{lstlisting}


\end{document}  